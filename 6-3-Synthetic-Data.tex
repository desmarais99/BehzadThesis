\Chapter{SYNTHETIC DATA GENERATION}\label{sec:Syn}


Simulated data has been playing an increasingly important role in EDM \citep{JEDM:baker2009}. The availability of new simulation software is contributing to this trend in every domain. As an example \citet{jafarpour2015quantifying} used simulated data for disease outbreak detection where simulated data is generated from a hypothesized model of a phenomena. The framework of this study relies on synthetic data. Getting back to the main contribution of this thesis, we proposed an approach for model selection that relies on synthetic data. Using synthetic data allows us to validate the sensitivity of the model selection approach to data specific parameters such as sample size and noise. Then the process of generating data should be accurate such that both data  and model specific parameters are applied properly. Every model studied here can be considered \textit{generative}, to the extent that they can generate synthetic data.

A unique advantage of synthetic data is that the model parameters can be predefined. Once the simulated data is generated with predefined parameters, a model can be trained over the generated data and a comparison with the original, known parameter values becomes possible. This may be the strongest benefit of synthetic data in assessing model fit. Since the hypothesis of this research relies on comparison the performance vector of real vs. synthetic data, then we can also consider data specific parameters of the real data in the generation process to make a better comparison of the results.

\section{Data generation parameters}

As described before, there exists two types of parameters for student test outcome:
\begin{itemize}
\item Model specific: These parameters are dependent on the ground truth of a dataset. Skills assessment models that were described in section \ref{SkillsAssessmentModels} considers different parameters. Depending on the type of student model, some of them may share a parameter such as Q-matrix for multiple skills models but still the interpretation of each model is unique. Generally an experiment should be designed to learn these parameters form a dataset or they can be generated under specific criteria. More details for each parameter is given later in this chapter.
\item Data specific (contextual): Regardless of the ground truth of a dataset, there are parameters that describe the dataste's specifications such as number of students. These parameters are common for any student model. They could possibly affect the performance of a model over a dataset.
\end{itemize}

Table \ref{tbl:param-DataGen} shows a complete list of both types of parameters required for data generation.

\newcommand{\tabitem}{~~\llap{\textbullet}~~}
\newcommand\VRule[1][\arrayrulewidth]{\vrule width #1}



\begin{table}
  \centering

\begin{tabular}{c|c|l!{\VRule[1.5pt]}l|l!{\VRule[1.5pt]}l|}
\multicolumn{3}{c}{}&\multicolumn{3}{c}{Parameters}\tabularnewline
\cline{4-6}
\multicolumn{3}{c!{\VRule[1.5pt]}}{Skills Model}&\multicolumn{2}{c!{\VRule[1.5pt]}}{Model specific}&\multicolumn{1}{c|}{Data specific}\tabularnewline
\cmidrule[1.5pt]{2-6}
&&NMF Conj. & & &  \tabularnewline
\cline{3-3}
&&NMF Add.&&\multirow{-2}{*}{ \tabitem  Q-matrix }&\multirow{-2}{*}{\tabitem Number of students } \tabularnewline
\cline{3-4}
&&DINA& \tabitem Slip & &\tabularnewline
\cline{3-3} 
&\multirow{-4}{*}{\begin{sideways} \scriptsize Multiple\end{sideways}}&DINO& \tabitem Guess&\multirow{-2}{*}{  \parbox[t]{3cm}{ \tabitem Students skills  \\ mastery matrix}}& \multirow{-2}{*}{\tabitem Number of items } \tabularnewline
\cmidrule[1.5pt]{2-5}
&&&\multicolumn{2}{l!{\VRule[1.5pt]}}{\tabitem Student ability  } &  \tabularnewline
&&&\multicolumn{2}{l!{\VRule[1.5pt]}}{\tabitem Item difficulty  } &\multirow{-2}{*}{\tabitem Number of skills } \tabularnewline
&&\multirow{-3}{*}{ IRT}&\multicolumn{2}{l!{\VRule[1.5pt]}}{\tabitem Item discrimination}  &\tabularnewline
\cline{3-5}
&&&\multicolumn{2}{l!{\VRule[1.5pt]}}{\tabitem Student Odds  } & \multirow{-2}{*}{\tabitem Test success rate }  \tabularnewline
&\multirow{-5}{*}{\begin{sideways} \scriptsize Single \end{sideways}}&\multirow{-2}{*}{ Expected}&\multicolumn{2}{l!{\VRule[1.5pt]}}{\tabitem Item Odds}  &  \tabularnewline
\cmidrule[1.5pt]{2-5}
&&&\multicolumn{2}{l!{\VRule[1.5pt]}}{\tabitem Initial Odds } & \multirow{-2}{*}{\tabitem Student score Variance} \tabularnewline
&&&\multicolumn{2}{l!{\VRule[1.5pt]}}{ \tabitem Odds ratio } & \tabularnewline
\multirow{-9}{*}{\begin{sideways} \scriptsize Contributed skills \end{sideways}}&\multirow{-3}{*}{\begin{sideways} \scriptsize Zero\end{sideways}}&\multirow{-3}{*}{ POKS }&\multicolumn{2}{l!{\VRule[1.5pt]}}{\tabitem PO structure}& \multirow{-2}{*}{\tabitem Item score Variance} \tabularnewline
\cmidrule[1.5pt]{2-6}
\end{tabular}
  \caption{Parameters involved in synthetic data generation}
  \label{tbl:param-DataGen}
\end{table}

\subsection{Assessing parameters}

To generate artificial data a set of parameters is required There are two approaches to assess required parameters to generate synthetic data:
\begin{itemize}
\item Parameters borrowed form a real data: Deriving contextual parameters from a real data is a straight forward task. To get model specific parameters we need to learn them from a real dataset given a model. In some cases there exists a predefined (or expert defined) parameter associated with the real data such as expert defined Q-matrices.
\item Generate parameters randomly based on a distribution: In the case that there exists no reference dataset to derive parameters we can randomly select some samples based on a distribution (for example normal distribution). Some parameters have specific conditions that if they become violated it changes the definition of the parameter. More details is given in next sections.

\end{itemize}

The rest of this chapter describes the data generation process based on each model assessment technique that is used in the proposed model selection approach.
% we can also specify which paramters has been borrowed from real data and which one has been generated

\section{POKS}
%first you have to get probabilities and which probability do you start with? where do they come from?

% explicitly say which distribution you used? and you can write them in a form of a equation(examples are given)
The method that generates synthetic dataset based on the POKS model requires a knowledge structure (KS). In this process the KS can be obtained from a real data set which allows us to make a better comparison of the results. It can also be generated as a random KS. The graphical demonstration of knowledge structure is an directed graph with no cycles allowed, except for symmetric relations.  

Other input parameters can make the random generation more specific,such as the number of links and number of independent trees in a KS\note{?? Needs to be defined.}. These parameters will change the item variance in the test result matrix. 

The generation of a random KS consists in randomly assigning 0-1 values to an upper triangular adjacency matrix. \annote{Once the KS adjacency matrix is created, we can assign values to each item that contributes with the initial odds of each node and odds ratio between a pair of nodes.}{not clear; odds ratio has not yet been introduced}

\note{Shouldn't you differenciate between the KS definition and the data generation?  Because KS can be taken from an existing data set, or generated at random with the procedure explained. Then comes the data generation part.}

%The second approach  is using probabilities for each node and a ratio for each pair for nodes which represent the strength of the link.  Based on the knowledge structure that has been picked we can also assess a set of initial probability for each node where a parent node gets a lower initial probability than its children. For inference in POKS framework to calculate the node's probability, we use standard Bayesian posteriors under the local independence assumption. The probability update for node $H$ given $E_1$,... $E_n$ can be written in following posterior odds form:

The approach for assigning values to each sample is using probabilities for each node and a ratio for each pair for nodes which represent the strength of the link.  Based on the knowledge structure that has been picked \note{picked out of what? it is either inferred from data or generated at random, no?} we can also assess a set of initial probability for each node where a parent node gets a lower initial probability than its children. Inference in the POKS framework to calculate the node's probability relies on standard Bayesian posteriors under the local independence assumption. The probability update for node $H$ given $E_1$,... $E_n$ can be written in following posterior odds form:
\note{This is not a clear description of the process.  What are the $E_i$ here? How is the first $E$ chosen and what is its initial probability? etc...}
\begin{equation}
O(H|E_1,E_2, ... , E_n) = O(H) \prod_{i}^{n} \frac{P(E_i|H)}{P(E_i | \overline{H})}
\label{EQPOKSratio}
\end{equation}
where odds definition is $O(H|E) = \frac{P(H|E)}{1-P(H|E)}$. If evidence $E_i$ is negative for observation $i$, then the ratio $\frac{P(\overline{E_i}|H)}{P(\overline{E_i}|\overline{H})}$ is used.

\note{Suggestion: (1) formally define all the parameters of the simulation, (2) explain how they are either derived from data, or initialized (presumably from a random distribution); (3) explain the algorithm for data point generation (should start with nodes that have no parents, then pick nodes for which all parents have values, etc.).  And what about nodes that have many parents?  This is the tricky question since POKS does not define multiple parents conditional probability tables---presumably this means that one parent is picked at random and that creates undeterminism of a kind, since picking one or the other parent would yield different probabilities.  And what about transitive relations?  In practice they are removed in the current POKS software if I recall correctly.  It would need to be explicitely mentioned.  Actually I see that the next paragraph addresses some of these issues.}

% if you don't have transitive relation in your process you need to state that please

All the parameters containing Partial Order structure, initial odds and odds ratio can also be obtained form a real dataset. For the case that these values are not predefined we need to assign random values with respect to the defined partial order structure. Once these values are defined, we can pick a node to sample with it's initial odds and consequently update it's neighbors' odds with equation~\ref{EQPOKSratio}. This process could be continued until there exists no node to sample.

% the updates are done based on deepth first or bearth first? how? describe clearly
\note{This is not clear.  Does it mean that after each new $E_i$ ``observed'', a nodes probability is updated?  For eg. Say we have this:
$a \rightarrow b, a \rightarrow c, b \rightarrow d, c \rightarrow d$.
if $a=1$ and $b$ gets updated, then $c$, will the probability of $d$ be updated from the posterior based on the observation of $a$ and $b$, or only one of them?  But presumably not based on $a$ as evidence also (if transitivity was removed.  And do you ensure that $d$ will not be assigned (observed) before both $c$ and $b$ are assigned?
}

The only way to control the total test result success rate and student/item score variance is to apply changes on the initial odds of nodes where we use for sampling student test result. For student scores variance, for each student the initial odds can be scaled such that the distribution of the initial odds stays the same for example we can double all the initial odds to represent a student with higher success rate. Changing the initial odds that follows a specific distribution will create a dataset in which the item variance is following that distribution. For overall successrate the initial odds can be scaled independent from students or item perspective, for example tripling all initial odds for all students will result in a higher success rate than the default values.

\section{IRT}

Equation~\ref{IRTEQ} shows the probability of a student given the ability of $\theta$ to success item $j$ which has the difficulty of $b_j$ and discrimination of $a_j$ based on IRT-2PL model. To generate a dataset that follows this model, we need to generate a sample of students with different abilities and items with different difficulties and discriminations. 
\begin{equation}
P(X_j\!=\!1\;|\;\theta) = \frac{1}{1+e^{-a_j(\theta-b_j)}}
\label{IRTEQ}
\end{equation}
\[-4 < \theta < 4\]
\[-3 < b_j < 3\]
\note{what about $a_j$?}

\note{Actually, it would be better to write this with equations, including the Poisson distribution.  See this paper (equation 2 and below): \url{http://educationaldatamining.org/EDM2011/wp-content/uploads/proc/edm2011_paper35_full_Desmarais.pdf}}

Students ability to answer questions and item difficulty are generated by a normal distribution with the mean of $0$ and standard deviation of $1.25$\note{why these numbers?  Shouldn't they be from the sample}. Discrimination (slope or correlation) representing how steeply the rate of success of individuals varies with their ability. In IRT-2PL, the values for discrimination are following a Poisson distribution with lambda parameter set to 10 that kept most values between 0.5 and 3. 

Item discrimination is a parameter that is learnt from training set. A perfect dataset that doesn't have any noise estimates this parameter with extremely large or small values (for example $300$ or $-300$)  which results in an unrealistic outcome prediction. For this purpose we add small amount of noise to prevent this condition.

\section{Linear NMF Conjunctive}

%Further studies with real data will be necessary, granted the results of the existing study warrants such work.

%We will consider that skills have a difficulty level. That difficulty will transfer to items that have this skill. The difficulty of the two-skills items will further increase by the fact that they require the conjunction of their skills. We 

%Examinees need to be assigned ability levels. The ability is, in fact, reflected by the set of skills mastered. , and the assigned ability level will increase this variance over the variance resulting from skill difficulty.

%Finally, two more parameters are used in the simulated the data, namely the $\mathit{slip}$ and $\mathit{guess}$ factors.  They are essentially noise factors and the greater they are, the more difficult is the task of inducing the Q-matrix from data.
%Skill difficulty and examinee ability are each expressed as a random normal variable. The probability density function of their sum provides the probability of mastery of the skill for the corresponding examinee. The skill vector is a sampling in \{0, 1\} based on each skill probability of mastery. 


The very first step to generate simulated test result for linear models is to define a Q-matrix that maps $k$ skills to $n$ items. The Q-matrix can be an expert predefined matrix or a random generated matrix. In the case of unavailable predefined Q-matrix, we defined a Q-matrix that provides all the possible combination of $k$ skills with a maximum of $Max$ skills per item, and at least one skill per item. A total of $\sum\limits_{k=1}^{Max} \dbinom{n}{k}$ items span this space of combinations for example $21$ items for 6 skills and maximum 2 skills per item. This matrix is shown in Figure~\ref{figqmatrixandResutM}. Items 1 to 15 are two-skills and items 16 to 21 are single-skill. Once the Q-matrix is created we can randomly replicate or eliminate some items to adjust the number of items to the desired number. 

\begin{figure}[ht]
\centering

\subfigure[Q-matrix of 6 skills]{
   \includegraphics[scale =0.5] {ExpectedQ.pdf}\label{PerfectQ}
 }\quad
 \subfigure[Synthetic data of 100 students with 10\% slip and 20\% guess factor]{
   \includegraphics[scale =0.5] {ResultM.pdf}
 }
\caption{Q-matrix and an example of simulated data with this matrix.  pale cells represent 1's and red ones represent 0's.}
\label{figqmatrixandResutM}
\end{figure}

There are two ways to apply item variance in the simulated data and both of them are based on manipulating the values of the Q-matrix. Applying skills difficulty on skills would transfer the difficulty to items that have this skill. The other method is to consider the same weight for skills difficulty but controlling the item variance by assigning different number of skills to items. For example items with 1 skill would become easier to answer comparing to items with two or more skills where skills difficulty is the same for all skills. 

The second step is to create a student skills mastery matrix which maps $k$ skills to $m$ students. In terms of ability for examinees we assigned random values to skills for students but  student variance show up as the variance in number of skills across examinees. At the same time we can apply the overall success rate on the skill mastery matrix using a threshold to discrete the assigned values in skills mastery. 

Once the Q-matrix and Skills mastery matrix is created we can produce the test result matrix with equation~(\ref{eq:6}). The last step is to add slip and guess factors which are set as constant values across items.

A sample of the results matrix is given in figure~\ref{figqmatrixandResutM} where pale cells represent a value of 1 and red cells are 0. Examinee ability shows up as vertical patterns, whereas skills difficulty creates horizontal patterns. As expected, the mean success rate of the 2-skills items 1 to 15 is lower than the single skill items 16 to 21.

\section{Linear NMF Additive}

The process to create synthetic data based on additive type of Q-matrix is almost the same as Conjunctive one. The difference is on the interpretation of the Q-matrix that changes the step where the result matrix is producing.

For this case each cell in the Q-matrix should be normalized on the bases of items. Although each skill has a specific weight to success an item but in our experiment we consider equal weight for all skills of an item. For this purpose all the values assigning to each item in Q-matrix should be divided by the number of involved skills for that item.

\begin{figure}[h]
\begin{tabular}{c}
\subfigure[{Additive Q-matrix\label{figNMFAddQM}}]{\begin{footnotesize}
$\begin{array}{ccc}
 && \textrm{items}\\

\mathrm{\begin{sideways}skills\end{sideways}} & & \left[\begin{array}{cccccccccccccccccc}

0.5&0.00&0.25&0.33&0.33&0.33&0.00&0.5&0.5&0.5&0.0&0.0&0.0&1&0&0&0\\
0.0&0.33&0.25&0.33&0.33&0.00&0.33&0.5&0.0&0.0&0.5&0.5&0.0&0&1&0&0\\
0.0&0.33&0.25&0.33&0.00&0.33&0.33&0.0&0.5&0.0&0.5&0.0&0.5&0&0&1&0\\
0.5&0.33&0.25&0.00&0.33&0.33&0.33&0.0&0.0&0.5&0.0&0.5&0.5&0&0&0&1


\end{array}\right]
\end{array}$
\end{footnotesize}
}\\
\begin{tabular}{cc}
\subfigure[Raw Result matrix ]{
   \includegraphics[scale =0.5] {NMFAddNonNoisy.pdf}\label{NMFAddNonNoisy}
 }\quad
&
\subfigure[Discretized result matrix with 20\% slip and 10\% guess factor]{
   \includegraphics[scale =0.5] {NMFAddNoisy}\label{NMFAddNoisy}
 }\quad
\end{tabular}
\end{tabular}
\caption{Additive model of Q-matrix and Corresponding synthetic data}
\label{figNMFAddgen}
\end{figure}

Figure~\ref{figNMFAddQM} shows an additive type of Q-matrix and figure~\ref{NMFAddNoisy} is the result of cross product of this matrix to a students skills mastery matrix. Since the model is additive, there are some pale cells and the paler a cell is, the more chance a student has to success the question. In conjunctive model the result matrix is either $0$ or $1$. 

\section{DINA/DINO}
These models are also categorized as linear models that use a Q-matrix. The Q-matrix can be predefined for a better comparison or can be synthetic. For synthetic Q-matrix we use the same method as described before. At the same time we can control the number of skills and items in generation of a Q-matrix.

Equation~\ref{DinoEQ3} requires three parameters to predict a test outcome. In our experiment we create a sequence of values for guess and slip in the range of $0$ and $0.2\%$. Examinee's skills can be generated by a normal distribution which should match the number of skills presented in the Q-matrix. The difference between creation of skills mastery matrix in DINA/DINO and NMF is the way that skills are appearing for each student. In DINA/DINO there is a predefined set of possible combination of skills that can be used in skills mastery for each student. For example for 3 skills, this set can have maximum 8 combination. There is a distribution that is assigned to this set which defines the probability of appearance for each combination in the skills mastery matrix. 
\begin{equation}
 P(X_{ij} \!=\! 1 \; | \; \xi_{ij}) \,=\, (1-s_j)^{\xi_{ij}} g_j^{1-\xi_{ij}}
\label{DinoEQ3}
\end{equation}
We can apply total success rate during the creation of student's skills mastery matrix where students skills define the success rate of a dataset. Since these models are behaving based on a single value that represents student ability, we need to calculate an array of abilities for each item given a set of skills for an student. In DINO we use a disjunction between skills of an item and skills of a student to determine whether the student has the ability or not where for DINA a conjunction applies.

