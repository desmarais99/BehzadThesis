%% -----------------------------------
%% ---> A MODIFIER PAR L'ETUDIANT <---
%% -----------------------------------
%%
%% Commandes qui affichent le titre du document, le nom de l'auteur, etc.
\newcommand\monTitre{Empirical means to validate skills models and assess the fit of a student model}
\newcommand\monPrenom{Behzad}
\newcommand\monNom{Beheshti}
\newcommand\monDepartement{génie informatique et génie logiciel}
\newcommand\maDiscipline{génie informatique}
\newcommand\monDiplome{D}        % (M)aîtrise ou (D)octorat
\newcommand\anneeDepot{2015}
\newcommand\moisDepot{DÉCEMBRE}
\newcommand\monSexe{M}           % "M" ou "F"
\newcommand\PageGarde{N}         % "O" ou "N"
\newcommand\AnnexesPresentes{O}  % "O" ou "N". Indique si le document comprend des annexes.
\newcommand\mesMotsClef{Educational data mining, Model selection, Model fitting, NMF, POKS, IRT, Synthetic data, Cross validation}
%%
%%  DEFINITION DU JURY
%%
%%  Pour la définition du jury, les macros suivantes sont definies:
%%  \PresidentJury, \DirecteurRecherche, \CoDirecteurRecherche, \MembreJury, \MembreExterneJury
%%
%%  Toutes les macros prennent 4 paramètres: Sexe (M/F), Prénom, Nom, Titres
\newcommand\monJury{\PresidentJury{M}{Yann-Ga\"{e}l}{Gu\'{e}h\'{e}neuc}{Doctorat}\\
\DirecteurRecherche{M}{Michel C.}{Desmarais}{Ph.~D.}\\
\MembreJury{M}{Michel}{Gagnon}{Ph.~D.}\\
\MembreExterneJury{M}{Zachary A.}{Pardos}{Ph.~D.}}

