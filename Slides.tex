\documentclass{beamer}
\setbeamertemplate{navigation symbols}{}
\usepackage{beamerthemeshadow}
\setbeamertemplate{footline}[text line]{%
  \parbox{\linewidth}{\vspace*{-8pt}\hfill\insertframenumber/\inserttotalframenumber}}
\usepackage{color}
\usepackage{rotating}
\usepackage{subfigure}
\usepackage[small,nohug,heads=vee]{diagrams}
\diagramstyle[labelstyle=\scriptstyle]
\definecolor{gray0}{rgb}{0.80,0.80,0.80}
\definecolor{gray1}{rgb}{0.60,0.60,0.60}
\definecolor{blue1}{rgb}{0.1,0.1,0.9}
\usepackage{verbatim} 
\usepackage{multirow}
\usepackage{natbib}
\usepackage[english,french]{babel}
\usepackage[utf8]{inputenc}
\usepackage{booktabs}
\usepackage{tikz}
\def\checkmark{\tikz\fill[scale=0.4](0,.35) -- (.25,0) -- (1,.7) -- (.25,.15) -- cycle;} 

%%%%%%%%%%
%%%%%%%%%%


\begin{document}
\title{Empirical means to validate skills models and assess the fit of a student model}  
\author{Behzad Beheshti\\{\footnotesize Supervisor: Michel C. Desmarais} }
\institute{{\tiny Génie Informatique Et Génie Logiciel\\ École Polytechnique de Montréal}}
\date{\today} 

\begin{frame}
\titlepage
\end{frame}


%%%%%%%%%%%%%%%%%%%%%%%%%%%


\section{Introduction}
\subsection{Problem Specification}
\begin{frame}\frametitle{Problem Specification}

\begin{itemize}
\item<1-> Student skills assessment models 
\item<2-> How to decide which are the most representative of the underlying ground truth? 
\item<3-> Static  Vs. Dynamic 
\item<4-> Model selection and goodness of fit 
\item<4-> A general answer : best performer
%\begin{itemize}
%\item<4-> Selecting a statistical model for a given data that is the best representative of the data.
%\item<4-> ``goodness of fit'' for a statistical model describes how well it fits a set of observation
%\end{itemize}
\item <5->Our contribution
\begin{itemize}
\item To make a comprehensive comparison of educational data model performances
\item To propose a new approach to assessing model fit
\end{itemize}
\item<6-> The proposed approach:
\begin{itemize}
\item Assessing the fit of the model to the underlying ground truth using a methodology based on \textbf{synthetic data}
\end{itemize}
\end{itemize}
\begin{overprint}
\vspace{-4.5cm}
\includegraphics<1>[trim= 0cm 6cm 0cm 12cm ,clip = true,scale =0.6]{images/Models}
\includegraphics<2>[trim= 0cm 5cm 0cm 6.5cm , scale =0.5]{images/Model-Selection}
\includegraphics<3>[trim= 0cm 5cm 0cm 5cm , scale =0.5]{images/Dynamic-Syn}
\includegraphics<4>[trim= 0cm 5cm 0cm 8.5cm , scale =0.5]{images/Model-Selected}
\includegraphics<5>[trim= 0cm 16cm 0cm 0cm , scale =0.5]{images/Models}
\includegraphics<6>[trim= 0cm 18cm 0cm 0cm , scale =0.5]{images/Models}
\end{overprint}
\end{frame}

\subsection{Introduction}
\begin{frame}
\vspace{-1cm}
    \begin{block}{Student skills assessment models}
    \vspace{-0.5cm}
\resizebox{.9\columnwidth}{4cm}{\includegraphics[trim=0.5cm 7cm 0.5cm 3cm,scale =0.4] {images/SkillsAssessments}}
    \end{block}
\begin{overprint}
      \onslide<1>
		\begin{columns}
		\begin{column}{0.35\textwidth}
      		\begin{itemize}
      		\vspace{-0.05cm}
      			\item Number of Skills
	      	\end{itemize}
	      \end{column}
	      \begin{column}{0.7\textwidth}
	      \end{column}
	     \end{columns}
      \onslide<2>
		\begin{columns}
		\begin{column}{0.35\textwidth}
      		\begin{itemize}
      		   	\vspace{-1.1cm}
      			\item Number of Skills
      			\item Q-matrix
	      	\end{itemize}
	      \end{column}
	      \begin{column}{0.7\textwidth}
	      
	      \resizebox{\columnwidth}{1.2cm}{
$\begin{array}{c|c|c}
   \begin{array}{cl}
   &\\
   &\\
   s_{1} : & \text{fraction multiplication}\\
   &\\
   s_{2} : & \text{fraction addition}\\
   &\\
   s_{3} : & \text{fraction reduction}\\
   &\\
   &
\end{array}
&
   \begin{array}{cc}
i_{1} & \frac{4}{\frac{12}{3}}+\frac{3}{5}=\frac{8}{5}\\
 &  \\
i_{2} & \frac{4}{\frac{12}{3}}=\frac{4{\times}3}{12}=\frac{12}{12}=1\\
 &  \\
i_{3} & 1+\frac{3}{5}=\frac{8}{5}\\
 &  \\
i_{4} & 2{\times}\frac{1}{2}=1\\
&
\end{array}
&
\begin{array}{c}
\begin{array}{cc}
 & \textrm{Skills}\\
 & \begin{array}{ccc}
s_{1} & s_{2} & s_{3}\end{array}\\
\mathrm{\begin{sideways}Items\end{sideways}}\begin{array}{c}
i_{1}\\
i_{2}\\
i_{3}\\
i_{4}
\end{array} & \left[\begin{array}{ccc}
1 & 1 & 1\\
1 & 0 & 1\\
0 & 1 & 1\\
1 & 0 & 1
\end{array}\right]
\end{array}\\
\\
\\

\end{array}
\end{array}$
	      }
	      
	      \end{column}
	     \end{columns}
      \onslide<3>
		\begin{columns}
		\begin{column}{0.35\textwidth}
      		\begin{itemize}
      			\item Number of Skills
      			\item Q-matrix 
      			\item Slip and Guess
	      	\end{itemize}
	      \end{column}
	      \begin{column}{0.7\textwidth}
	      \end{column}
	     \end{columns}
	     \onslide<4>
	     		\begin{columns}
		\begin{column}{0.4\textwidth}
      		\begin{itemize}
      		      		   	\vspace{-0.5cm}
      			\item Number of Skills
      			\item \textbf{Q-matrices (types)}
      			\item Slip and Guess
	      	\end{itemize}
	      \end{column}
	      \begin{column}{0.25\textwidth}
	      \begin{enumerate}
	      \item Conjunctive
	      \item Additive 
	      \item Disjunctive
	      \end{enumerate}
	      \end{column}
	      \begin{column}{0.4\textwidth}
	      
	      \resizebox{0.66\columnwidth}{1.2cm}{

$\begin{array}{c}
\begin{array}{cc}
 & \textrm{Skills}\\
 & \begin{array}{ccc}
s_{1} & s_{2} & s_{3}\end{array}\\
\mathrm{\begin{sideways}Items\end{sideways}}\begin{array}{c}
i_{1}\\
i_{2}\\
i_{3}\\
i_{4}
\end{array} & \left[\begin{array}{ccc}
1 & 1 & 1\\
1 & 0 & 1\\
0 & 1 & 1\\
1 & 0 & 1
\end{array}\right]
\end{array}\\
\\
\\

\end{array}$
	      }
	      
	      \end{column}
	     \end{columns}
\end{overprint}
\end{frame}

\newcommand{\tabitem}{~~\llap{\textbullet}~~}
\newcommand\VRule[1][\arrayrulewidth]{\vrule width #1}


\subsection{Predictive performance}
\begin{frame}
\begin{itemize}
\item<1-> \textbf{Performance of a model} over a data set
\item<2-> Model parameters
\item<3-> Performance vector
\item<3-> Performance signature
\item<4-> Performance prototype
\item<5-> Target performance vector
\end{itemize}
\begin{overprint}
\vspace{-2cm}
\onslide<1> \centering \includegraphics[trim=0cm 8.5cm 2.4cm 2.4cm,clip=true,width=\textwidth]{images/Methodology.pdf}
\onslide<2> \vspace{1cm} \centering \includegraphics[width=.7\textwidth]{images/Parameters-Model}
\onslide<3>
		\begin{columns}
		\begin{column}{0.5\textwidth}
		 %\includegraphics[trim=0cm 0cm 0cm 0cm,clip=true,width=\textwidth]{images/Performance-Presentation}	   
		 \vspace{1.5cm}   
   \includegraphics[trim=0cm 0cm 0cm 1.5cm,clip=true,scale =0.55] {images/Predictive-Preformace_Sig.pdf}
   \includegraphics[trim=0cm 0cm 0cm 1.5cm,clip=true,scale =0.55] {images/Predictive-Preformace_Base.pdf}

		 \end{column}
	      \begin{column}{0.5\textwidth}
	      \[\begin{array}{l|c}
	      Model & Performace\\
	      \cline{1-2}
			Expected & 0.72\%\\
			POKS & 0.80\%\\
			IRT & 0.74\%\\
			NMF.Conj & 0.94\%\\
			Dina & 0.99\%\\
			NMF.Add & 0.60\%\\
			Dino & 0.65\%
			\end{array}
			\]
	      \end{column}
	     \end{columns}
\onslide<4>\vspace{3cm} The \textit{performance vector} associated with the synthetic data of a model class.
\onslide<5>\vspace{3cm} The \textit{performance vector} of the real data set to classify.
\end{overprint}
\end{frame}

\section{Main contributions}
\subsection{Experiment 1: Predictive performance}
\begin{frame}\frametitle{Research questions}
\begin{enumerate}
\item \checkmark What is the \textit{performance vector} of student skills assessment models over real and over synthetic data created using the same models?
\begin{itemize}
\item Experiment 1: Predictive performance of models over real and synthetic data sets
\end{itemize}
\end{enumerate}
\end{frame}


\begin{frame}\frametitle{Datasets}
\centering
\resizebox{7cm}{!}{
\centering
\footnotesize
\begin{tabular}{|l|c|c|r|r|l|}
\hline

%\rowcolor{\color[rgb]{.8,.8,.8}}
\multirow{2}{*}{Data set} & \multicolumn{3}{c|}{Number of} & {\parbox{6ex}{\center Mean\\Score}} & \multirow{2}{*}{Q-matrix}\tabularnewline
\cline{2-4} 
%\rowcolor{\color[rgb]{.8,.8,.8}}
 & Skills & Items & Students &  & \tabularnewline
\hline
\hline
%\rowcolor{\color[rgb]{.9,.9,.9}}
\multicolumn{6}{|c|}{\textit{Synthetic}}\\
\hline
\hline
1.Random & 7 & 30 & 700 &0.75& $\mathbf{Q}_{01}$\tabularnewline
\hline
2.POKS & 7 & 20 & 500 &0.50 & $\mathbf{Q}_{02}$\tabularnewline
\hline
3.IRT-2PL & 5 & 20 & 600 &0.50& $\mathbf{Q}_{03}$\tabularnewline
\hline
4.DINA & 7 & 28 & 500 &0.31& $\mathbf{Q}_5$\tabularnewline
\hline
5.DINO & 7 & 28 & 500 &0.69& $\mathbf{Q}_6$\tabularnewline
\hline
\multicolumn{6}{|l|}{Linear (Matrix factorization)}\\
\hline
6.~~~Conj. & 8 & 20 & 500 &0.24& $\mathbf{Q}_1$\tabularnewline
\hline
7.~~~Comp. & 8 & 20 & 500 &0.57& $\mathbf{Q}_1$ \tabularnewline
\hline
\hline
%\rowcolor{\color[rgb]{.9,.9,.9}}
\multicolumn{6}{|c|}{\textit{Real}}\\
\hline
\hline
8.Fraction & 8 & 20 & 536 &0.53& $\mathbf{Q}_1$\tabularnewline
\hline
9.Vomlel & 6 & 20 & 149 &0.61& $\mathbf{Q}_4$\tabularnewline
\hline
10.ECPE & 3 & 28 & 2922 &0.71& $\mathbf{Q}_3$\tabularnewline
\hline
\multicolumn{6}{|l|}{Fraction subsets and variants of $\mathbf{Q}_{1}$}\\
\hline
11.~~~1 & 5 & 15 & 536 &0.53& $\mathbf{Q}_{10}$\tabularnewline
\hline
12.~~~2/1 & 3 & 11 & 536 &0.51& $\mathbf{Q}_{11}$\tabularnewline
\hline
13.~~~2/2 & 5 & 11 & 536 &0.51& $\mathbf{Q}_{12}$\tabularnewline
\hline
14.~~~2/3 & 3 & 11 & 536 &0.51& $\mathbf{Q}_{13}$\tabularnewline
\hline
\end{tabular}}
\end{frame}

\begin{frame}\frametitle{Predictive performance of models over synthetic datasets}
\vspace{-0.5cm}
\includegraphics[scale =0.45] {images/Syn}
\begin{overprint}
      \onslide<1> \small The highest performance is for the generative model behind the dataset
%The highest performance is for the skills assessment technique that is the same as the generative model behind the dataset%%%%the generative model behind the data set is the same as the skills assessment technique, the corresponding technique’s performance is the best, or close to the best.
	  \onslide<2> \small Data sets have unique pattern of performance vector across models
     \onslide<3> The random data set has a flat performance across techniques 
     %%%which corresponds to the dominant class prediction. 
	  \onslide<4> The capacity of recognizing a data set’s true model relies on this uniqueness characteristic
\end{overprint}
\end{frame}

\begin{frame}\frametitle{Predictive performance of models over real datasets}
\vspace{-0.5cm}
\includegraphics[scale =0.4] {images/Real}
\begin{overprint}
      \onslide<1> In most cases, the best performer is close to the baseline
      \onslide<2> The pattern of the Fraction performance data set repeats over its subsets %Fraction-1, Fraction-2/1 and Fraction-2/2, in spite of the different number of skills and different subsets of questions
      %this similarity among Fraction data set and its derivative suggests that in spite of the model differences (different Q-matrices and item subsets), the performance signature remains constant across these data sets.
      \onslide<3> None of the real data sets show the large the amplitude and the differences found in the synthetic data sets models %Scles are 4 times bigger
\end{overprint}
\end{frame}

\begin{frame}\frametitle{Vector space of accuracy performances}
\begin{itemize}
\item Performance vectors of datasets in columns(Data points in the performance space)
\end{itemize}
\resizebox{\columnwidth}{!}{
\begin{tabular}{lccccccc}
  \toprule
  \multicolumn{1}{c}{\multirow{2}{*}{\textbf{Model}}} & \multicolumn{7}{c}{\textbf{Synthetic data set }} \\
  \cline{2-8}
  & \multicolumn{1}{c}{{\textit{Random}}} & \multicolumn{1}{c}{{POKS}} & \multicolumn{1}{c}{{IRT}} & \multicolumn{1}{c}{{DINA}} & \multicolumn{1}{c}{{DINO}} & \multicolumn{1}{c}{{Linear .Conj}} & \multicolumn{1}{c}{{Linear .Comp}} \\ 
  \hline
  \textit{Expected} & \textbf{0.75} & 0.91 & 0.90 & 0.72 & 0.72 & 0.78 & 0.93 \\ 
  POKS & 0.75 & \textbf{0.94} & 0.94 & 0.81 & 0.81 & 0.90 & 0.94 \\ 
  IRT & 0.75 & 0.91 & \textbf{0.95} & 0.73 & 0.73 & 0.79 & 0.89 \\ 
  DINA & 0.75 & 0.77 & 0.81 & \textbf{1.00} & 0.65 & \textbf{0.98} & 0.89 \\ 
  DINO & 0.75 & 0.63 & 0.56 & 0.66 & \textbf{1.00} & 0.68 & 0.91 \\ 
  NMF.Conj & 0.75& 0.59 & 0.53 & 0.95 & 0.65 & 0.97 & 0.58 \\ 
  NMF.Comp & 0.75 & 0.76 & 0.79 & 0.59 & 0.93 & 0.70 & \textbf{0.98} \\ 
  \bottomrule
\end{tabular}}
\vspace{.7cm}

The diagonal generally displays the best performance %the diagonal (in bold face, except for one, corresponding to the match between the underlying synthetic model and the model performance) generally displays the best perfor- mance since it corresponds to the alignment of the model and the ground truth behind the data.

\end{frame}

\subsection{Experiment 2: Sensitivity of the Model performance}
\begin{frame}\frametitle{Research questions}
\begin{enumerate}
\item \checkmark What is the \textit{performance vector} of student skills assessment models over real and over synthetic data created using the same models?
\begin{itemize}
\item Experiment 1: Predictive performance of models over real and synthetic data sets
\end{itemize}
\item \textbf{Is the \textit{performance vector} unique to each synthetic data type (data from the same ground truth model)?}
\begin{itemize}
\item Experiment 2: Sensitivity of the Model performance over different data generation parameters
\end{itemize}
Are they stable in addition to be unique.
\end{enumerate}
\end{frame}

\begin{frame}\frametitle{Variation of sample size over synthetic data sets}
\includegraphics[scale =0.37] {images/SampleSize}
\begin{overprint}
      \onslide<1> Obviously, the signature pattern did not change significantly for some parameters such as \textbf{Sample size}.
\end{overprint}
\end{frame}

\begin{frame}\frametitle{Variation of number of items over synthetic data sets}
\includegraphics[scale =0.37] {images/numberofitems}
\begin{overprint}
	  \onslide<1> Even for synthetic data the ground truth should not necessarily be close to 100%.
      \onslide<2> The Performance signature shifts down once the number of items degrades.
\end{overprint}
\end{frame}

\begin{frame}\frametitle{Data specific parameters}
\begin{enumerate}
\item Sample size (Number of students)
\item Number of items
\item Number of latent skills
\item Item score variance 
\item Student score variance
\item Average success rate\pause
\end{enumerate}
Conclusion :
\begin{itemize}
      \item Contextual factors can potentially influence the performance of a model%the best performer may not be the model that is most representative of the ground truth, but instead it may be the result of contextual factors that make this model outperform the ground truth one.
	  \item For better comparison of the results, we can also consider \textbf{data specific parameters} of the real data in the generation process%we can also consider data specific parameters of the real data in the generation process to make a better comparison of the results.
\end{itemize}
\end{frame}

\subsection{Experiment 3: Signature Approach}
\begin{frame}\frametitle{Research questions}
\begin{enumerate}
\item \checkmark What is the \textit{performance vector} of student skills assessment models over real and over synthetic data created using the same models?
\begin{itemize}
\item Experiment 1: Predictive performance of models over real and synthetic data sets
\end{itemize}
\item \checkmark Is the \textit{performance vector} unique to each synthetic data type (data from the same ground truth model)?
\begin{itemize}
\item Experiment 2: Sensitivity of the Model performance over different data generation parameters
\end{itemize}
\item \textbf{Can the \textit{performance vector} be used to define a method to reliably identify the ground truth behind the synthetic data?}
\begin{itemize}
\item Experiment 3: Model selection based on performance vector classification
\end{itemize}
\end{enumerate}
\end{frame}

\begin{frame}\frametitle{Signature framework}
This approach relies on generating synthetic datasets
\begin{overprint}
\onslide<1> 
		\begin{columns}
		\begin{column}{0.7\textwidth}
		\vspace{-0.8cm}
			\includegraphics[trim=1cm 10cm 1cm 1cm,scale=0.43]{images/Approach1.pdf}
		\end{column}
		\begin{column}{0.3\textwidth}
		\end{column}
		\end{columns}
\onslide<2> 		\begin{columns}
		\begin{column}{0.7\textwidth}
		\vspace{-0.8cm}
			\includegraphics[trim=1cm 10cm 1cm 1cm,scale=0.43]{images/Approach2.pdf}
		\end{column}
		\begin{column}{0.3\textwidth}
		\end{column}
		\end{columns}
		\onslide<3> 		\begin{columns}
		\begin{column}{0.7\textwidth}
			\vspace{-0.8cm}
			\includegraphics[trim=1cm 10cm 1cm 1cm,scale=0.43]{images/Approach3.pdf}
		\end{column}
		\begin{column}{0.3\textwidth}
			\hspace{1.5cm}Recall
          
          \vspace{1cm}
           \rightline{\includegraphics[trim=0cm 0cm 0cm 1.5cm,clip=true,scale =0.5] {images/Predictive-Preformace_Sig.pdf}}


		\end{column}
		\end{columns}
		\onslide<4> 		\begin{columns}
		\begin{column}{0.7\textwidth}
			\vspace{-0.8cm}			
			\includegraphics[trim=1cm 10cm 1cm 1cm,scale=0.43]{images/Approach4.pdf}
		\end{column}
		\begin{column}{0.3\textwidth}
		\end{column}
		\end{columns}\end{overprint}
\end{frame}

\begin{frame}\frametitle{Pool of synthetic datasets}
\includegraphics[trim=1cm 17cm 1cm 1cm,scale=0.55]{images/Data-Gen-Break-Down.pdf}
\begin{overprint}
\onslide<1> There exists 6 skills assessment models 
\onslide<2> There exists 6 skills assessment models X 6 data specific parameters 
\onslide<3> There exists 6 skills assessment models X 6 data specific parameters~X 4 values 
\onslide<4> There exists 6 skills assessment models X 6 data specific parameters~X 4 values X 10 samples = 1440 samples in the pool
\end{overprint}
\end{frame}

\begin{frame}\frametitle{Degree of similarity between six synthetic datasets based on the correlation}
\resizebox{\columnwidth}{!}{\begin{tabular}{c|c|c|c|c|c|c|c|}
\multicolumn{2}{c}{} & \multicolumn{6}{c}{Synthetic Datasets} \tabularnewline
\multicolumn{8}{c}{} \tabularnewline
\cline{3-8} 
\multicolumn{2}{c|}{} & POKS & IRT & NMF Conj. & DINA & NMF Add. & DINO\tabularnewline
\cline{2-8}
\cline{2-3}
&POKS & \textbf {0.96} & \multicolumn{1}{|c}{} & \multicolumn{1}{c}{} & \multicolumn{1}{c}{} & \multicolumn{1}{c}{}\tabularnewline
\cline{2-4}
&IRT & 0.86 & \textbf {0.96} & \multicolumn{1}{|c}{} & \multicolumn{1}{c}{} & \multicolumn{1}{c}{} & \multicolumn{1}{c}{}\tabularnewline
\cline{2-5}
&NMF Conj. & 0.22 & -0.20 & \textbf {0.96} & \multicolumn{1}{|c}{} & \multicolumn{1}{c}{} & \multicolumn{1}{c}{}\tabularnewline
\cline{2-6}
&DINA & 0.02 & -0.40 & 0.94 & \textbf {0.96} & \multicolumn{1}{|c}{} & \multicolumn{1}{c}{}\tabularnewline
\cline{2-7}
&NMF Add. & 0.44 & 0.75 & -0.62 & -0.73 & \textbf {0.93} & \multicolumn{1}{|c}{}\tabularnewline
\cline{2-8}
\multicolumn{1}{c|}{\multirow{-6}{*}{\begin{sideways}Synthetic Datasets\end{sideways}}}&DINO & -0.15 & 0.20 & -0.70 & -0.69 & 0.63 & \textbf {0.95}\tabularnewline
\cline{2-8}
\end{tabular}}
\begin{overprint}
      \onslide<1> Item1 Item1 Item1 Item1 Item1 Item1 Item1 Item1 Item1 Item1 Item1 Item1 Item1 Item1 Item1 Item1 Item1 Item1 Item1 Item1 Item1 Item1 Item1 Item1 
\end{overprint}
\end{frame}

\begin{frame}\frametitle{Degree of similarity between six synthetic datasets and the ground truth based on the correlation}
\resizebox{\columnwidth}{!}{\begin{tabular}{c|c|c|c|c|c|c|c|c|}

\multicolumn{2}{c}{}&\multicolumn{7}{c}{Real Datasets}\tabularnewline   
\multicolumn{9}{c}{}\tabularnewline   
\cline{6-9}
\multicolumn{5}{c|}{}&\multicolumn{4}{c|}{Fraction subsets}   \tabularnewline   
\cline{3-9} 
\multicolumn{2}{c|}{}   & Vomlel &ECPE &Fraction &1&21&22&23\tabularnewline
\cline{2-9}
\cline{2-9}
&Random & 0.58 &\textbf {0.73} & 0.61   & 0.43 & 0.24 & 0.61 & 0.57 \tabularnewline
\cline{2-9}
&IRT & \textbf {0.90} & 0.42 & 0.72   & 0.88 & 0.60 & 0.77 & 0.61 \tabularnewline
\cline{2-9}
&DINA & -0.38  & -0.09 &   0.23 &   0.30 & 0.56 & 0.06 & 0.38 \tabularnewline
\cline{2-9}
&DINO & 0.34 & 0.15  &  -0.18 &  -0.31 & 0.10 & -0.08 & 0.38 \tabularnewline
\cline{2-9}
&POKS & 0.75 &0.40  &  \textbf {0.83}  &  \textbf {0.95} &\textbf {0.70} & \textbf {0.83} & \textbf {0.80}\tabularnewline
\cline{2-9}
 &NMF Conj. & -0.05 & 0.54  & 0.51   & 0.55  & 0.66 & 0.33 & 0.57\tabularnewline
\cline{2-9}
\multicolumn{1}{c|}{\multirow{-7}{*}{\begin{sideways}Synthetic Datasets\end{sideways}}}&NMF Add. & 0.39 &0.06   & -0.04   & -0.19 & -0.03 & 0.13 & 0.28\tabularnewline
\cline{2-9}
\end{tabular}
}
\begin{overprint}
      \onslide<1> Item1 Item1 Item1 Item1 Item1 Item1 Item1 Item1 Item1 Item1 Item1 Item1 Item1 Item1 Item1 Item1 Item1 Item1 Item1 Item1 Item1 Item1 Item1 Item1 
\end{overprint}
\end{frame}

\subsection{Experiment 4: Signature vs. best performer}
\begin{frame}\frametitle{Research questions}
\begin{enumerate}
\item \checkmark What is the \textit{performance vector} of student skills assessment models over real and over synthetic data created using the same models?
\begin{itemize}
\item Experiment 1: Predictive performance of models over real and synthetic data sets
\end{itemize}
\item \checkmark Is the \textit{performance vector} unique to each synthetic data type (data from the same ground truth model)?
\begin{itemize}
\item Experiment 2: Sensitivity of the Model performance over different data generation parameters
\end{itemize}
\item \checkmark Can the \textit{performance vector} be used to define a method to reliably identify the ground truth behind the synthetic data?
\begin{itemize}
\item Experiment 3: Model selection based on performance vector classification
\end{itemize}
\item \textbf{How does the method compare with the standard practice of using the model with the best performance?}
\begin{itemize}
\item Experiment 4: Signature vs. best performer classification
\end{itemize}
\end{enumerate}
\end{frame}

\section{Conclusion}
\begin{frame}\frametitle{Research questions}
\begin{enumerate}
\item \checkmark What is the \textit{performance vector} of student skills assessment models over real and over synthetic data created using the same models?
\begin{itemize}
\item Experiment 1: Predictive performance of models over real and synthetic data sets
\end{itemize}
\item \checkmark Is the \textit{performance vector} unique to each synthetic data type (data from the same ground truth model)?
\begin{itemize}
\item Experiment 2: Sensitivity of the Model performance over different data generation parameters
\end{itemize}
\item \checkmark Can the \textit{performance vector} be used to define a method to reliably identify the ground truth behind the synthetic data?
\begin{itemize}
\item Experiment 3: Model selection based on performance vector classification
\end{itemize}
\item \checkmark How does the method compare with the standard practice of using the model with the best performance?
\begin{itemize}
\item Experiment 4: Signature vs. best performer classification
\end{itemize}
\end{enumerate}

\end{frame}

\subsection{Questions}
\begin{frame}\frametitle{Thank you}
\begin{figure}
\includegraphics[scale =0.55] {images/Question} 
\end{figure}
\end{frame}
\end{document}
