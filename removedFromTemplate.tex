
Texte en \emph{italique}, mot \mbox{insécable}.\\
Texte \ul{souligné}, \hl{surligné}, \textbf{gras}.\\
Texte entre ``guillemets''.\\
Police \texttt{monospace}.\\
Un mot courant en réseautique mobile: n\oe{}ud\footnote{Note de bas de page.}.\\
L'objet RSVP \texttt{SENDER\_TEMPLATE}.\\
Nom d'un auteur: \citeauthor{RFC_IPv4}.\\
Un architecture de 32~bits.\\

\section{Plan du mémoire}  % 0.5 page
Un tableau:
\begin{table}[htbp]
  \centering
  \caption{Constantes et variables du modèle analytique}
  \begin{tabular}{|c|l|}
    \hline\rowcolor[gray]{0.8}\color{black}
    Symbole         & Description\\\hline
    $\lambda$       & Taux d'arrivée moyen des requêtes de réservation de ressources\\\hline
    $\frac{1}{\mu}$ & Durée moyenne d'une session\\\hline
    $C$             & Capacité d'une cellule (nombre de sessions supportées)\\\hline
    $v_{moy}$       & Vitesse moyenne des MN dans le réseau d'accès\\\hline
    $L$             & Longueur d'un côté d'une cellule carrée\\\hline
    $n$             & Nombre moyen de MN dans une cellule\\\hline
    $\rho$          & Charge d'une cellule\\\hline
    $P_b$           & Probabilité de blocage d'une requête de réservation\\\hline
    $P_f$           & Probabilité d'interruption forcée d'une session\\\hline
    $P_c$           & Probabilité de compléter une session avec succès\\\hline
    $\Delta{}T$     & Délai de transmission\\\hline
  \end{tabular}
  \label{tab:Definitions}
\end{table}

La formule d'\mbox{Erlang-B}:
\begin{equation}
  P_b = \frac{\frac{\rho^C}{C!}}{\sum\limits_{x=0}^{C}\frac{\rho^x}{x!}}
  \label{eq:Pblock}
\end{equation}

Une autre équation:
\begin{equation}
  \begin{split}
    P_c &= (1 - P_b) \times (1 -  P_f)^N\\
        &= (1 - P_b)^{N+1}
  \end{split}
  \label{eq:ProbComplete}
\end{equation}

Enfin, l'expression suivante indique le moment à partir duquel les
réservations de ressources sont en force:
\begin{equation}
  \Delta{}T_{init} =
  \begin{cases}
    2\Delta{}T_{E2E} & \Delta{}T_{wan} > (\Delta{}T_{rad} + \Delta{}T_{net})\\
    \Delta{}T_{E2E} + 3(\Delta{}T_{rad} + \Delta{}T_{net}) & \text{sinon}
  \end{cases}
  \label{eq:InitCost}
\end{equation}

\clearpage

Une courbe:

\begin{figure}[htb]
\centering
\includegraphics[width=5in]{LinkUsage}
\caption{Probabilité de blocage}
\label{fig:BlockProb}
\end{figure}

\begin{itemize}
\item la fiabilité (acheminement des données avec succès)~;
\item le délai de \mbox{bout-en-bout} de la source vers la destination~;
\item la variation du délai de \mbox{bout-en-bout} (\emph{jitter})~;
\item la bande passante requise (le débit des informations).
\end{itemize}

\paragraph{Le niveau paragraphe} est plus bas encore dans la hiérarchie\ldots
Citation entre parenthèses \citep[voir][]{ART_ARTP}.

\clearpage

%%
%% ELEMENTS DE LA PROBLEMATIQUE
%%
\section{Éléments de la problématique}  % environ 3 pages
La description de \mbox{l'en-tête} commun de RSVP est détaillée ci-dessous:\\
\begin{tabular}{p{1in}p{4.5in}}
&\\ % Ligne vide
\texttt{Ver}: & \texttt{4 bits}\\
          & Version du protocole. La version actuelle est~1.\\[5pt]
\texttt{Flags}: & \texttt{4 bits}\\
          & Aucun Flag n'est défini. L'émetteur doit (\textbf{MUST})
          mettre le champ à zéro et le récepteur doit (\textbf{MUST})
          ignorer ce champ.\\[5pt]
\texttt{Msg Type}: & \texttt{8 bits}\\
          & Type de message\\[5pt]
\texttt{Checksum}: & \texttt{16 bits}\\
          & Complément à un du complément à un de la somme des champs
          de \mbox{l'en-tête}, avec le champ Checksum à~0 pour des
          fins de calcul. La valeur~0 signifie qu'aucun Checksum n'a
          été transmis. Si le résultat du calcul du Checksum donne~0,
          la valeur 0xFFFF doit être stockée dans ce champ.\\[5pt]
\texttt{TTL}: & \texttt{8 bits}\\
          & Valeur originelle du champ \texttt{TTL} utilisée pour
          transmettre ce message.\\[5pt]
\texttt{Reserved}: & \texttt{8 bits}\\
          & Réservé pour usage futur. L'émetteur doit (\textbf{MUST})
          mettre le champ à zéro et le récepteur doit (\textbf{MUST})
          ignorer ce champ.\\[5pt]
\texttt{Length}: & \texttt{16 bits}\\
          & Longueur totale du message en octets, incluant
          \mbox{l'en-tête} commun et tous les objets de longueur
          variable.
\end{tabular}

\subsection{Autres types de structures de données}
L'énumération:
\begin{enumerate}
\item Un item~;
\item Un autre item.
\end{enumerate}


\subsection{Le protocole IPv6}
Voir la Figure~\ref{fig:IPv6} pour plus de détails. Le champs DSCP est
décrit dans le Tableau~\ref{tab:RangesDSCP}.

\begin{figure}[htb]
\centering
\includegraphics[width=4in]{IPv6_header}
\caption{L'en-tête IPv6}
\label{fig:IPv6}
\end{figure}

\begin{table}[ht]
\caption{Plages de valeurs pour le champ \texttt{DSCP}}
\centering
\begin{tabular}{|c|c|l|}
\hline\rowcolor[gray]{0.8}\color{black}
Plage & Valeurs & Règle d'assignation\\\hline
1 & xxxxx0 & Assignation par une norme de l'IANA\\\hline
2 & xxxx11 & Expérimentation/Usage local\\\hline
3 & xxxx01 & Expérimentation/Usage local (pourrait être jointe à la plage 1)\\\hline
\end{tabular}
\label{tab:RangesDSCP}
\end{table}

% On veut éviter que la figure et le tableau soient placés au-delà de la section courante.
\FloatBarrier
