\chapter*{RÉSUMÉ}\thispagestyle{headings}
\addcontentsline{toc}{compteur}{RÉSUMÉ}
Dans l'exploration de données en éducation, ou dans l'exploration de données en général, les analystes qui souhaitent construire une classification ou un modèle de régression sur les données nouvelles et inconnues sont confrontés à un très grand nombre de choix. Les techniques d'apprentissage automatique offrent de nos jours la possibilité d'apprendre et de former des modèles de données de tailles et variétés toujours plus grandes. Parallèlement à cette augmentation de la variété des modèles qui peuvent être définis et formés à partir des données, vient la question abordée dans cette thèse: comment décider lesquels sont plus représentatifs de la réalité sous-jacente.

La pratique courante est de former différents modèles, et d'utiliser celui permettant la meilleure prédiction comme le meilleur modèle. Toutefois, la performance du modèle varie généralement avec des facteurs tels que la taille de l'échantillon, la variable ciblée et l'entropie du prédicteur, le bruit, les valeurs manquantes, etc. Par exemple, la capacité d'adaptation d'un modèle au bruit et sa capacité à faire face à la petite taille de l'échantillon peut donner de meilleures performances que le modèle sous-jacent pour un ensemble de données.

Par conséquent, le meilleur modèle peut ne pas être le plus représentatif de la réalité, mais peut être le résultat de facteurs contextuels qui rendent celui-ci meilleur que le modèle sous-jacent.

Nous étudions la question de l'évaluation de modèles différents à partir de données synthétiques en définissant un espace vectoriel des performances de ceux-ci, et nous utilisons une approche de plus proches voisins avec une distance de corrélation pour identifier le modèle sous-jacent. Cette approche est basée sur les définitions et les procédures suivantes. Soit un ensemble de modèles, $\mathcal{M}$, et un vecteur~$\mathbf{p}$ de longueur~$|\mathcal{M}|$ qui contient la performance de chaque modèle sur un ensemble de données. Ce vecteur représente un point qui caractérise l'ensemble de données dans l'espace de performance. Pour chaque modèle $M$ dans~$\mathcal {M}$, nous déterminons un nouveau point dans l'espace de performance qui correspond à des données synthétiques générées par le modèle~$M$. Puis, pour un ensemble de données, nous trouvons le point synthétique le plus proche, en utilisant la corrélation comme distance, et considérons le modèle l'ayant généré comme le modèle sous-jacent.

Les résultats montrent que, pour les ensembles de données synthétiques, leurs ensembles de modèles sous-jacents sont généralement plus souvent correctement identifiés par l'approche proposée plutôt que par le modèle avec la meilleure performance. Ils montrent aussi que les modèles sémantiquement similaires sont également plus rapprochés dans l'espace de performance que les modèles qui sont basés sur des concepts très différents.